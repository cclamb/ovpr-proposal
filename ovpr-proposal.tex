\documentclass[12pt,letterpaper]{article}
\usepackage[utf8]{inputenc}
\usepackage{amsmath}
\usepackage{amsfonts}
\usepackage{amssymb}
\usepackage{graphicx}
\usepackage{fourier}
\usepackage{fullpage}
\author{Christopher C. Lamb, Gregory L. Heileman}
\title{OVPR Equipment Funding Request}
\date{}
\begin{document}

\maketitle

\abstract{
The Informatics Group affiliated with the Electrical and Computer Engineering Department has a small cloud laboratory that we have been building for the past three years. We have spent initial funding, and need to replace some of the older systems as well as acquire additional systems to both effectively teach students affiliated with the laboratory and to compete for research grants.
}

\section{Need}
Need

\newpage
\section{Budget Summary}
Budget

HP 3500-48G-PoE+ yl Switch
(J9311A) 

KVM

HP TFT7600 G2 KVM Console Rackmount Keyboard US Monitor \$1699

Cables

10 HP servers HP ProLiant DL360p Gen8 E5-2670v2 2P 32GB-R P420i/1GB 8 SFF 750W RPS Svr/S-Buy
(748301-S01)



\newpage
\section{Principal Investigators}
{\bf Gregory L. Heileman} (heileman@unm.edu) serves as the Associate Provost for Curriculum at the University of New Mexico (UNM), a position he has held since 2011.  He received the BA degree from Wake Forest University in 1982, the MS degree in Biomedical Engineering and Mathematics from the University of North Carolina-Chapel Hill in 1986, and the Ph.D. degree in Computer Engineering from the University of Central Florida in 1989.  In 1990 he joined the Department of Electrical and Computer Engineering (ECE) at UNM, where he is currently a Professor. From 2005-2011 he served as ECE associate chair (director of undergraduate programs), and led the department through two ABET accreditation visits.  In 2011 he became an ABET program evaluator. 

{\bf Christopher C. Lamb} (cclamb@unm.edu) currently serves as an Cyber-security Research Scientist with Sandia National Laboratories (SNL) and as a Research Assistant Professor affiliated with the Electrical and Computer Engineering department at the University of New Mexico (UNM). He currently participates in a wide range of research roles at SNL, ranging from analyzing moving target defense controls to designing neural algorithms for custom hardware implementation. At UNM, he leads students in research projects focusing on the security of software defined networking and cloud systems as well as academic analytic visualization and secure drone video capture and analysis. He also has extensive experience designing and developing mission-critical distributed systems for a wide range of government departments and agencies. Prior to joining SNL, Dr. Lamb served in executive roles and as a principal consultant for a variety of technology companies in the southwest. Dr. Lamb has a B.S. in Mechanical Engineering from New Mexico State University, an M.S. in Computer Science from the University of New Mexico, as well as a Ph.D. in Computer Engineering with a focus on Computational Intelligence from the University of New Mexico. He is a The Open Group Architecture Framework (TOGAF) 9 Certified Enterprise Architect and a Certified Information Systems Security Professional (CISSP) through the International Information Systems Security Certification Consortium.

\end{document}